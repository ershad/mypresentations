\documentclass[14pt]{beamer}


\usepackage{fancyvrb,relsize}
\usepackage{slashbox}
\usetheme{CambridgeUS}
\usepackage{amsmath}
\usepackage{courier}
\usepackage{enumerate}
\setbeamertemplate{itemize items}[circle]
\usepackage{graphicx}
\usepackage{listings}
\usepackage{fancybox}
\usepackage[normalem]{ulem}

\mode<presentation>
{
\setbeamertemplate{footline}
{\rightline{\insertframenumber/\inserttotalframenumber}}
}

\title{Some Fundamental Doubts...}
\author{\textbf{Raghesh A}}
\date{November 4th, 2011}

\begin{document}

% slide
\begin{frame}
\titlepage
\end{frame}

% slide
\begin{frame}{The Doubts}
\begin{itemize}
\item Why Vim/Emacs, Not Notepad/gedit
\pause
\item Doubts/Misunderstandings in C
\pause
  \begin{itemize}
    \item Does \textcolor{red}{\#include $<$stdio.h$>$} contain printf?
\pause
    \item What is the basic job of a data type?
\pause
    \item Can we compare \emph{printf} and functions in maths like $f(x) = x^2 + 1$ ?
\pause
    \item Signed and Unsigned, Why 2's complement?
\pause
    \item Why ./a.out? Why not a.out?
\pause
    \item "I should run my program in DOS. There is no segmentation fault". Should I?
  \end{itemize}
\pause
\item Why latex, not MS Word?
\end{itemize}
\end{frame}

%slide
\begin{frame}{Functions in Maths and Programming Language. Are they similar}
        \textbf{printf("hello", var1, var2)}
\begin{itemize}
\item $f(x,y) = x^2 + y^2$
\pause
\item What is the domain and range of this the function $f$?
\item What is the domain and range of the function printf?
\pause
\end{itemize}
\end{frame}

%slide
\begin{frame}{Signed and Unsigned, Why 2's complement?}
\begin{itemize}
\item Assumption: There are only 4 bits available
\pause
\item Unsigned : 0 to 15(0x0000 to 0x1111)
\item Signed : 0 to 7(0x0000 to 0x0111) -8 to -1(1000 to 1111)
\pause
\end{itemize}
\end{frame}

%slide
\begin{frame}{The Doubts...}
\begin{itemize}
\item Are {\textbf {we}} students capable of contributing?
  \begin{itemize}
\pause
    \item We need an email id
    \item Basic programming knowledge
    \item A clean mind
\pause
    \item Aim Google Summer of Code(GSOC).
  \end{itemize}
\end{itemize}
\end{frame}

% slide
\begin{frame}{GSOC.. Climbing steps...}
\begin{itemize}
\item Acquire necessary background knowledge
\pause
\item Pick up a free software project
\pause
\item Learn the project
\pause
\item Building Confidence - Fix a bug OR Implement a small feature
\pause
\item Start contributing and become a developer
\pause
\item Apply for GSOC(Prepare a good Application)
\pause
\item Be proud of being a GSOCer
\pause
\item Work hard for the summer
\pause
\item Enjoy the Reward
\pause
\item Post GSOC - Don't stop:Continue contributing
\pause
\end{itemize}
\end{frame}

\end{document}
